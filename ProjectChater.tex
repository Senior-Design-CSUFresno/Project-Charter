%Written by: Aaron Stillmaker
%October 23, 2018
%ECE 186A - Senior Design
%
%This is a template you can use for your Project Charter, though you will need to change a good portion of it for your own needs.

\documentclass[12pt,onecolumn]{IEEEtran}			%Everything is default, except  which is journal, 10pt font, and final draft

%add packages here:

\title{ \hfill  \vspace{2in} \\Draft Project Charter Instructions \vspace{1in} }	%the \vspace is moving the title down from the top of the page
\author{California State University, Fresno \\
Electrical and Computer Engineering Department \\
ECE 186A - Senior Design I \\ 					%I am jamming all of the stuff I want on the title page into the Author spot, I know, this isn't elegant
Fall 2018 - Dr. Stillmaker \\ 					%Note the \\ line returns to make sure to put each one on a new line
\vspace{12pt} 								%I put in some space to get the due date listed a little lower
\textbf{Due:} Friday, October 26, 2018 by 10:00 AM to Blackboard \\ 
\vspace{2in}								%You may need to mess with this space, this space is above the signature lines.


\vspace{4in}}								%You will need to mess with this space, this space is after the signature lines.  The idea
										%is to make the title page by itself.  I know, this isn't elegant either, but IEEE formatting doesn't
										%make title pages

\begin{document}


\maketitle									%This generates the title from the information given above.
\thispagestyle{empty}						%This removes the page number from the title page.
\newpage


 \IEEEPARstart{Y}{ou} 						%This makes the cool looking formatting for the first word.  It isn't required.
 and your team should generate a single PDF document using \LaTeX \hspace{1pt} and the \\ \verb3\documentclass[12pt,onecolumn]{IEEEtran}3.
 You are welcome to use the posted \\ \verb1ProjectCharter.tex1 file posted on Blackboard, and just modify that to add your
 own information, and meet the requirements.
 I don't want to give a definite page limit, but I am expecting this document to be somewhere around fifteen to thirty pages, depending on your number of figures and tables, though making sure to 
 adequately cover all the requested
 material is what matters.  In other words, I expect that at this point you have rather thoroughly thought through your project, and should
 have a lot to say about it.
 
 Below I list the major parts that should be included.  Keep in mind that you will likely have subsections
 under these sections, which I won't directly specify, but you should logically organize your report. \\
 
 \textbf{It should contain the following parts:}
 
\section*{Title Page}
 Your title page should include your project's name, that it is a project charter, the university name, the department name,
 ECE 186A, Fall 2018, the date, the instructor's name (me), your technical advisor's name, and all of your ECE group member's names.
 After this, you may need to add more space
 to make sure it is only the above mentioned information that was included.
 
 \section*{Revision History}
 You should have a page that explains the revisions.  It should have a version number, who made the changes, when the version was finalized,
 and what were the high level changes made.  At this point you might or might not have multiple versions, if for example your technical
 advisor suggested changes that you had to go back and perform.  This shouldn't be so fine-grained that you log every time you make a slight
 change.  A good rule of thumb is you should have a new version any time the report is given outside of your immediate group, this way
 if you give a version to your technical advisor for example, you can keep straight what version they have.
 
 \section*{Abstract}
 A brief explanation of your project, in about one or two paragraphs, and must be less than one page.
 
 \section*{Table of Contents}
 You should have a table of contents, which can be generated by a \verb1\tableofcontents1 command.
 
 \section*{List of Figures}
 You should have a list of figures, which can be generated by a \verb1\listoffigures1 command.
 
 \section*{List of Tables}
 You should have a list of tables, which can be generated by a \verb1\listoftables1 command.
 
 \section{Introduction}
Clearly state your project, and what is the main deliverable.  List your team members, and what their specific responsibilities will
be in completing the project.  This section should have any details about your team member's strengths, but just very simply what
their responsibilities are in about a sentence
 (i.e. Student One is responsible for being the project manager, micro-controller communications programming, 
 and design of PCB to interface with components).
 
 This is where you should list your project purpose and justification.
 
 \section{Project Objectives and Success Criteria}
 Here you should list your project objectives, likely in a bulleted form, which clearly states the objectives accomplished by your project.
 You should also list your success criteria here.  As mentioned in class, make sure you include both ambitious success criteria, as well
 as more easily attainable success criteria.  This is how you will measure how successful you were in your project.
 
 \section{High-Level Requirements}
 A list of requirements at a high level.  This will include things such as being implementable in 2 semesters, completed within a budget, as well
 as other requirements placed on your project.
 
\section{Assumptions, Constraints, and Standards}
Describe the background, strengths and weaknesses, of each
team member's knowledge, as it pertains to the project.  Explain in detail what background information is going to need to be researched
and learned to complete your project.  You should list specific courses that you have taken during your time at Fresno State that you will need to 
use information from to complete your project.  Note, for a successful Senior Design Topic, you must be using a breadth of knowledge.

At this point you should have completed some extensive background research, which should be included here.

List any other constraints you may have.

Include a \textbf{list of the standards that you will be following in this project}.  Remember that there is an
IEEE standard for just about any communication protocol, format, or language used in ECE, so make sure to
look it up if you are unsure.  This is required.
 
\section{Project Description and Boundaries}
Describe your project in detail in this section.  List out all of the major components, as well as all of the minor components underneath that.
Explain all the major difficulties with the project.  
List who your technical faculty advisor is, what their area of expertise is, and how
you will use that expertise in your project.

Here is where you should \textbf{include your block diagrams, flow charts, and circuit diagrams}.
Each project should have multiple parts that need to communicate, so you need a diagram that shows the details
of the names of the connections, the size of the busses, etc.  Those with programs, which I also believe is all,
should have outlines of flowcharts for the programs that will be used in your project.

\section{High-Level Risks}
What are the risks associated with your project?  For most, there are power systems you must be concerned about,
or moving parts, or a number of other possibilities.  List what these are, and how you are addressing keeping
your project development safe.

\section{Milestone Schedule}
This should have a list of the milestones, who is in charge of each milestone, and when it should be accomplished.

\section{Gantt Chart}
Include your updated Gantt chart for this and next semesters.  Note, it is expected that there can be some
overlap between the Gantt and the schedule, but they focus on different things, and present the data
in a different way.

\section{Equipment and Budget}
This should have a detailed list of what equipment you will be using, both to build you project, as well as to
test your project.  This includes items you have, need to purchase, or need to borrow.

You should also have a detailed budget of what components you will be purchasing.

\section{Roles of Team Members}
 Clearly describe each team member, what their strengths are, what their weaknesses are, and what their role is
 for the project.
 
 \section{Stakeholder List}
 List all of the stakeholders for your project, including why they are a stakeholder.
 This should include team members (ECE as well as ME), the instructor (me), your technical advisor, your sponsors (if any),
 and anyone else that has a stake in the project being successfully completed.
 
 \section{Project Approval Requirements}
 List what is required for your project to meet the approval of you various stakeholders.
 
 \section{Approvals}
 This is a list of signatures of all of the stakeholders.
 
%\underline{\hspace{3in}} \\
%\textbf{Project Manager:} Student One}			%Uncomment this section and fill in your team names and your technical advisor's name.
%\underline{\hspace{3in}} \\					%You are welcome to have more than one technical advisor if you wish..
%\textbf{Team Member:} Student Two}
%\underline{\hspace{3in}} \\
%\textbf{Team Member:} Student Three}
%\underline{\hspace{3in}} \\
%\textbf{Technical Advisor:} Dr. Esteemed Professor}

\section*{References}
You will use Bibtex to generate references for all of the material you cited, using \verb1\cite{}1, to write this report.
Use the \verb1\bibliographystyle{IEEEtran}1 to generate your references.

Remember, you need to compile your .tex source file once as Latex, then once as Bibtex, then once again as Latex.
The first compile determines what the citations are, the second generates the list of references, the third assigns
the correct reference numbers.
 
\end{document}
